% Options for packages loaded elsewhere
\PassOptionsToPackage{unicode}{hyperref}
\PassOptionsToPackage{hyphens}{url}
\PassOptionsToPackage{dvipsnames,svgnames,x11names}{xcolor}
%
\documentclass[
  11pt,
]{article}

\usepackage{amsmath,amssymb}
\usepackage{iftex}
\ifPDFTeX
  \usepackage[T1]{fontenc}
  \usepackage[utf8]{inputenc}
  \usepackage{textcomp} % provide euro and other symbols
\else % if luatex or xetex
  \usepackage{unicode-math}
  \defaultfontfeatures{Scale=MatchLowercase}
  \defaultfontfeatures[\rmfamily]{Ligatures=TeX,Scale=1}
\fi
\usepackage{lmodern}
\ifPDFTeX\else  
    % xetex/luatex font selection
\fi
% Use upquote if available, for straight quotes in verbatim environments
\IfFileExists{upquote.sty}{\usepackage{upquote}}{}
\IfFileExists{microtype.sty}{% use microtype if available
  \usepackage[]{microtype}
  \UseMicrotypeSet[protrusion]{basicmath} % disable protrusion for tt fonts
}{}
\makeatletter
\@ifundefined{KOMAClassName}{% if non-KOMA class
  \IfFileExists{parskip.sty}{%
    \usepackage{parskip}
  }{% else
    \setlength{\parindent}{0pt}
    \setlength{\parskip}{6pt plus 2pt minus 1pt}}
}{% if KOMA class
  \KOMAoptions{parskip=half}}
\makeatother
\usepackage{xcolor}
\usepackage[margin=0.75in]{geometry}
\setlength{\emergencystretch}{3em} % prevent overfull lines
\setcounter{secnumdepth}{5}
% Make \paragraph and \subparagraph free-standing
\makeatletter
\ifx\paragraph\undefined\else
  \let\oldparagraph\paragraph
  \renewcommand{\paragraph}{
    \@ifstar
      \xxxParagraphStar
      \xxxParagraphNoStar
  }
  \newcommand{\xxxParagraphStar}[1]{\oldparagraph*{#1}\mbox{}}
  \newcommand{\xxxParagraphNoStar}[1]{\oldparagraph{#1}\mbox{}}
\fi
\ifx\subparagraph\undefined\else
  \let\oldsubparagraph\subparagraph
  \renewcommand{\subparagraph}{
    \@ifstar
      \xxxSubParagraphStar
      \xxxSubParagraphNoStar
  }
  \newcommand{\xxxSubParagraphStar}[1]{\oldsubparagraph*{#1}\mbox{}}
  \newcommand{\xxxSubParagraphNoStar}[1]{\oldsubparagraph{#1}\mbox{}}
\fi
\makeatother

\usepackage{color}
\usepackage{fancyvrb}
\newcommand{\VerbBar}{|}
\newcommand{\VERB}{\Verb[commandchars=\\\{\}]}
\DefineVerbatimEnvironment{Highlighting}{Verbatim}{commandchars=\\\{\}}
% Add ',fontsize=\small' for more characters per line
\usepackage{framed}
\definecolor{shadecolor}{RGB}{241,243,245}
\newenvironment{Shaded}{\begin{snugshade}}{\end{snugshade}}
\newcommand{\AlertTok}[1]{\textcolor[rgb]{0.68,0.00,0.00}{#1}}
\newcommand{\AnnotationTok}[1]{\textcolor[rgb]{0.37,0.37,0.37}{#1}}
\newcommand{\AttributeTok}[1]{\textcolor[rgb]{0.40,0.45,0.13}{#1}}
\newcommand{\BaseNTok}[1]{\textcolor[rgb]{0.68,0.00,0.00}{#1}}
\newcommand{\BuiltInTok}[1]{\textcolor[rgb]{0.00,0.23,0.31}{#1}}
\newcommand{\CharTok}[1]{\textcolor[rgb]{0.13,0.47,0.30}{#1}}
\newcommand{\CommentTok}[1]{\textcolor[rgb]{0.37,0.37,0.37}{#1}}
\newcommand{\CommentVarTok}[1]{\textcolor[rgb]{0.37,0.37,0.37}{\textit{#1}}}
\newcommand{\ConstantTok}[1]{\textcolor[rgb]{0.56,0.35,0.01}{#1}}
\newcommand{\ControlFlowTok}[1]{\textcolor[rgb]{0.00,0.23,0.31}{\textbf{#1}}}
\newcommand{\DataTypeTok}[1]{\textcolor[rgb]{0.68,0.00,0.00}{#1}}
\newcommand{\DecValTok}[1]{\textcolor[rgb]{0.68,0.00,0.00}{#1}}
\newcommand{\DocumentationTok}[1]{\textcolor[rgb]{0.37,0.37,0.37}{\textit{#1}}}
\newcommand{\ErrorTok}[1]{\textcolor[rgb]{0.68,0.00,0.00}{#1}}
\newcommand{\ExtensionTok}[1]{\textcolor[rgb]{0.00,0.23,0.31}{#1}}
\newcommand{\FloatTok}[1]{\textcolor[rgb]{0.68,0.00,0.00}{#1}}
\newcommand{\FunctionTok}[1]{\textcolor[rgb]{0.28,0.35,0.67}{#1}}
\newcommand{\ImportTok}[1]{\textcolor[rgb]{0.00,0.46,0.62}{#1}}
\newcommand{\InformationTok}[1]{\textcolor[rgb]{0.37,0.37,0.37}{#1}}
\newcommand{\KeywordTok}[1]{\textcolor[rgb]{0.00,0.23,0.31}{\textbf{#1}}}
\newcommand{\NormalTok}[1]{\textcolor[rgb]{0.00,0.23,0.31}{#1}}
\newcommand{\OperatorTok}[1]{\textcolor[rgb]{0.37,0.37,0.37}{#1}}
\newcommand{\OtherTok}[1]{\textcolor[rgb]{0.00,0.23,0.31}{#1}}
\newcommand{\PreprocessorTok}[1]{\textcolor[rgb]{0.68,0.00,0.00}{#1}}
\newcommand{\RegionMarkerTok}[1]{\textcolor[rgb]{0.00,0.23,0.31}{#1}}
\newcommand{\SpecialCharTok}[1]{\textcolor[rgb]{0.37,0.37,0.37}{#1}}
\newcommand{\SpecialStringTok}[1]{\textcolor[rgb]{0.13,0.47,0.30}{#1}}
\newcommand{\StringTok}[1]{\textcolor[rgb]{0.13,0.47,0.30}{#1}}
\newcommand{\VariableTok}[1]{\textcolor[rgb]{0.07,0.07,0.07}{#1}}
\newcommand{\VerbatimStringTok}[1]{\textcolor[rgb]{0.13,0.47,0.30}{#1}}
\newcommand{\WarningTok}[1]{\textcolor[rgb]{0.37,0.37,0.37}{\textit{#1}}}

\providecommand{\tightlist}{%
  \setlength{\itemsep}{0pt}\setlength{\parskip}{0pt}}\usepackage{longtable,booktabs,array}
\usepackage{calc} % for calculating minipage widths
% Correct order of tables after \paragraph or \subparagraph
\usepackage{etoolbox}
\makeatletter
\patchcmd\longtable{\par}{\if@noskipsec\mbox{}\fi\par}{}{}
\makeatother
% Allow footnotes in longtable head/foot
\IfFileExists{footnotehyper.sty}{\usepackage{footnotehyper}}{\usepackage{footnote}}
\makesavenoteenv{longtable}
\usepackage{graphicx}
\makeatletter
\def\maxwidth{\ifdim\Gin@nat@width>\linewidth\linewidth\else\Gin@nat@width\fi}
\def\maxheight{\ifdim\Gin@nat@height>\textheight\textheight\else\Gin@nat@height\fi}
\makeatother
% Scale images if necessary, so that they will not overflow the page
% margins by default, and it is still possible to overwrite the defaults
% using explicit options in \includegraphics[width, height, ...]{}
\setkeys{Gin}{width=\maxwidth,height=\maxheight,keepaspectratio}
% Set default figure placement to htbp
\makeatletter
\def\fps@figure{htbp}
\makeatother

\makeatletter
\@ifpackageloaded{caption}{}{\usepackage{caption}}
\AtBeginDocument{%
\ifdefined\contentsname
  \renewcommand*\contentsname{Table of contents}
\else
  \newcommand\contentsname{Table of contents}
\fi
\ifdefined\listfigurename
  \renewcommand*\listfigurename{List of Figures}
\else
  \newcommand\listfigurename{List of Figures}
\fi
\ifdefined\listtablename
  \renewcommand*\listtablename{List of Tables}
\else
  \newcommand\listtablename{List of Tables}
\fi
\ifdefined\figurename
  \renewcommand*\figurename{Figure}
\else
  \newcommand\figurename{Figure}
\fi
\ifdefined\tablename
  \renewcommand*\tablename{Table}
\else
  \newcommand\tablename{Table}
\fi
}
\@ifpackageloaded{float}{}{\usepackage{float}}
\floatstyle{ruled}
\@ifundefined{c@chapter}{\newfloat{codelisting}{h}{lop}}{\newfloat{codelisting}{h}{lop}[chapter]}
\floatname{codelisting}{Listing}
\newcommand*\listoflistings{\listof{codelisting}{List of Listings}}
\makeatother
\makeatletter
\makeatother
\makeatletter
\@ifpackageloaded{caption}{}{\usepackage{caption}}
\@ifpackageloaded{subcaption}{}{\usepackage{subcaption}}
\makeatother

\ifLuaTeX
  \usepackage{selnolig}  % disable illegal ligatures
\fi
\usepackage{bookmark}

\IfFileExists{xurl.sty}{\usepackage{xurl}}{} % add URL line breaks if available
\urlstyle{same} % disable monospaced font for URLs
\hypersetup{
  pdftitle={Project 2 Helper Guide: Cluster Analysis Workflow},
  colorlinks=true,
  linkcolor={blue},
  filecolor={Maroon},
  citecolor={Blue},
  urlcolor={Blue},
  pdfcreator={LaTeX via pandoc}}


\title{Project 2 Helper Guide: Cluster Analysis Workflow}
\author{}
\date{}

\begin{document}
\maketitle


\section*{Overview}\label{overview}
\addcontentsline{toc}{section}{Overview}

This guide will help you plan and run your cluster analysis for Project
2. Use it to:

\begin{itemize}
\tightlist
\item
  Choose a \emph{clear} and \emph{specific} research question
\item
  Select and prepare relevant variables
\item
  Clean and summarize your data
\item
  Use cluster analysis to segment your data
\item
  Interpret your results
\end{itemize}

\begin{center}\rule{0.5\linewidth}{0.5pt}\end{center}

\section*{Step 1: Choose a Question You Want to
Answer}\label{step-1-choose-a-question-you-want-to-answer}
\addcontentsline{toc}{section}{Step 1: Choose a Question You Want to
Answer}

Think about \textbf{who} or \textbf{what} you want to cluster.

\begin{itemize}
\item
  Are you clustering \textbf{customers}, \textbf{stores}, or
  \textbf{products}? \vspace{2em}
\item
  What is the goal of your analysis? (e.g., segmenting high vs.~low
  activity stores, grouping customers by shopping behavior, high-selling
  vs.~low-selling brands) \vspace{2em}
\end{itemize}

\textbf{Write your question here:}

\begin{quote}
\emph{My research question is:}
\end{quote}

\vspace{2em}

\begin{quote}
\emph{What types of {[}X{]} exist based on {[}Y{]}?} (``X'' = the thing
you are clustering (store, customer, or product segments); ``Y'' =
spending habits, product diversity, visit frequency, etc.)
\end{quote}

\newpage

\section*{Step 2: Identify Your Unit of
Analysis}\label{step-2-identify-your-unit-of-analysis}
\addcontentsline{toc}{section}{Step 2: Identify Your Unit of Analysis}

What are the rows in your final dataset?

\begin{quote}
\emph{I am clustering:}
\end{quote}

\begin{itemize}
\tightlist
\item[$\square$]
  Stores\\
\item[$\square$]
  Customers\\
\item[$\square$]
  Products\\
\item[$\square$]
  Something else: \_\_\_\_\_\_\_\_\_\_\_\_\_\_\_\_\_\_\_\_
\end{itemize}

\section*{Step 3: Join and Clean Your
Data}\label{step-3-join-and-clean-your-data}
\addcontentsline{toc}{section}{Step 3: Join and Clean Your Data}

Join the datasets you need to build the variables for your clustering
question.

\begin{itemize}
\tightlist
\item
  shopper\_info\\
\item
  store\_info\\
\item
  gtin
\end{itemize}

\begin{quote}
\emph{What is my unit of analysis? (from Step 2)}
\end{quote}

\vspace{2em}

\begin{quote}
\emph{What tables will I join and why?}
\end{quote}

\vspace{3em}

\textbf{Filter or transform your data:}

\begin{itemize}
\tightlist
\item
  Remove invalid prices (e.g., \texttt{unit\_price\ \textless{}=\ 0})
\item
  Keep or remove \texttt{gtin\ =\ NA} and \texttt{gtin\ =\ 0} (fuel)
  depending on your question
\item
  Create total spending: \texttt{unit\_price\ *\ unit\_quantity}
\end{itemize}

\begin{quote}
\emph{What cleaning steps did you take and why?}
\end{quote}

\newpage

\section*{Step 4: Build Your Summary
Dataset}\label{step-4-build-your-summary-dataset}
\addcontentsline{toc}{section}{Step 4: Build Your Summary Dataset}

Now summarize your data to create one row per unit (store, customer, or
product).

\vspace{2em}

\textbf{Write down 3--5 variables that might be useful for clustering:}

\begin{longtable}[]{@{}
  >{\raggedright\arraybackslash}p{(\columnwidth - 4\tabcolsep) * \real{0.2532}}
  >{\raggedright\arraybackslash}p{(\columnwidth - 4\tabcolsep) * \real{0.5190}}
  >{\raggedright\arraybackslash}p{(\columnwidth - 4\tabcolsep) * \real{0.2278}}@{}}
\toprule\noalign{}
\begin{minipage}[b]{\linewidth}\raggedright
Variable Name
\end{minipage} & \begin{minipage}[b]{\linewidth}\raggedright
What It Measures
\end{minipage} & \begin{minipage}[b]{\linewidth}\raggedright
Notes / Formula
\end{minipage} \\
\midrule\noalign{}
\endhead
\bottomrule\noalign{}
\endlastfoot
& & \\
& & \\
& & \\
& & \\
& & \\
\end{longtable}

\begin{quote}
Are these numeric variables?
\end{quote}

\vspace{2em}

\begin{quote}
Are they all available in the data?
\end{quote}

\vspace{2em}

\begin{quote}
Do you need to aggregate them (e.g., total sales per store, average
items per visit)?
\end{quote}

\newpage

\section*{Step 5: Explore Your
Variables}\label{step-5-explore-your-variables}
\addcontentsline{toc}{section}{Step 5: Explore Your Variables}

Use \texttt{ggpairs()} to check for skew and correlation.

\begin{Shaded}
\begin{Highlighting}[]
\FunctionTok{library}\NormalTok{(GGally)}

\NormalTok{your\_data }\SpecialCharTok{\%\textgreater{}\%}
  \FunctionTok{select}\NormalTok{(var1, var2, var3, ...) }\SpecialCharTok{\%\textgreater{}\%}
  \FunctionTok{ggpairs}\NormalTok{()}
\end{Highlighting}
\end{Shaded}

\begin{quote}
What do you notice?
\end{quote}

\begin{itemize}
\tightlist
\item
  Are any variables very skewed?
\end{itemize}

\vspace{2em}

\begin{itemize}
\tightlist
\item
  Are any variables highly correlated?
\end{itemize}

\vspace{2em}

\begin{itemize}
\tightlist
\item
  Do any variables need to be log-transformed?
\end{itemize}

\vspace{2em}

\begin{quote}
Make a list of any variables you want to log-transform:
\end{quote}

\newpage

\section*{Step 6: Transform and Scale Your
Data}\label{step-6-transform-and-scale-your-data}
\addcontentsline{toc}{section}{Step 6: Transform and Scale Your Data}

\subsection*{Log-transform skewed variables (add +1 to avoid
log(0))}\label{log-transform-skewed-variables-add-1-to-avoid-log0}
\addcontentsline{toc}{subsection}{Log-transform skewed variables (add +1
to avoid log(0))}

\begin{Shaded}
\begin{Highlighting}[]
\NormalTok{your\_data }\OtherTok{\textless{}{-}}\NormalTok{ your\_data }\SpecialCharTok{\%\textgreater{}\%}
  \FunctionTok{mutate}\NormalTok{(}
    \AttributeTok{log\_var1 =} \FunctionTok{log}\NormalTok{(var1 }\SpecialCharTok{+} \DecValTok{1}\NormalTok{),}
    \AttributeTok{log\_var2 =} \FunctionTok{log}\NormalTok{(var2 }\SpecialCharTok{+} \DecValTok{1}\NormalTok{)}
\NormalTok{  )}

\DocumentationTok{\#\# Create the dataset you\textquotesingle{}ll use for clustering \{.unnumbered\}}
\NormalTok{cluster\_data }\OtherTok{\textless{}{-}}\NormalTok{ your\_data }\SpecialCharTok{\%\textgreater{}\%}
  \FunctionTok{select}\NormalTok{(log\_var1, log\_var2, var3, ...) }\SpecialCharTok{\%\textgreater{}\%}
  \FunctionTok{drop\_na}\NormalTok{()}
\end{Highlighting}
\end{Shaded}

\subsection*{Scale the data}\label{scale-the-data}
\addcontentsline{toc}{subsection}{Scale the data}

\begin{Shaded}
\begin{Highlighting}[]
\NormalTok{cluster\_scaled }\OtherTok{\textless{}{-}} \FunctionTok{scale}\NormalTok{(cluster\_data)}
\end{Highlighting}
\end{Shaded}

\begin{quote}
Which variables are in your scaled clustering data?
\end{quote}

\vspace{2em}

\section*{Step 7: Determine the Number of
Clusters}\label{step-7-determine-the-number-of-clusters}
\addcontentsline{toc}{section}{Step 7: Determine the Number of Clusters}

Use these plots to decide on an optimal number of clusters:

\begin{Shaded}
\begin{Highlighting}[]
\FunctionTok{fviz\_nbclust}\NormalTok{(cluster\_scaled, kmeans, }\AttributeTok{method =} \StringTok{"wss"}\NormalTok{)}
\FunctionTok{fviz\_nbclust}\NormalTok{(cluster\_scaled, kmeans, }\AttributeTok{method =} \StringTok{"silhouette"}\NormalTok{)}
\end{Highlighting}
\end{Shaded}

\begin{quote}
What does the elbow plot suggest?
\end{quote}

\vspace{2em}

\begin{quote}
What does the silhouette plot suggest?
\end{quote}

\vspace{2em}

\begin{quote}
How many clusters will you use? Why?
\end{quote}

\newpage

\section*{Step 8: Run the Clustering
Algorithm}\label{step-8-run-the-clustering-algorithm}
\addcontentsline{toc}{section}{Step 8: Run the Clustering Algorithm}

\begin{Shaded}
\begin{Highlighting}[]
\FunctionTok{set.seed}\NormalTok{(}\DecValTok{123}\NormalTok{)}
\NormalTok{kmeans\_fit }\OtherTok{\textless{}{-}} \FunctionTok{kmeans}\NormalTok{(cluster\_scaled, }\AttributeTok{centers =}\NormalTok{ X, }\AttributeTok{nstart =} \DecValTok{25}\NormalTok{)}

\NormalTok{final\_clusters }\OtherTok{\textless{}{-}}\NormalTok{ your\_data }\SpecialCharTok{\%\textgreater{}\%}
  \FunctionTok{mutate}\NormalTok{(}\AttributeTok{cluster =}\NormalTok{ kmeans\_fit}\SpecialCharTok{$}\NormalTok{cluster)}
\end{Highlighting}
\end{Shaded}

\begin{Shaded}
\begin{Highlighting}[]
\NormalTok{final\_clusters }\SpecialCharTok{\%\textgreater{}\%}
  \FunctionTok{group\_by}\NormalTok{(cluster) }\SpecialCharTok{\%\textgreater{}\%}
  \FunctionTok{summarise}\NormalTok{(}\FunctionTok{across}\NormalTok{(}\FunctionTok{everything}\NormalTok{(), mean))}
\end{Highlighting}
\end{Shaded}

\begin{quote}
How many observations are in each cluster?
\end{quote}

\vspace{2em}

\begin{quote}
What patterns do you notice when you group by cluster?
\end{quote}

\vspace{2em}

\section*{Step 9: Interpret and Visualize
Clusters}\label{step-9-interpret-and-visualize-clusters}
\addcontentsline{toc}{section}{Step 9: Interpret and Visualize Clusters}

Now use your original variables (not log-transformed) to summarize each
cluster.

\begin{quote}
What makes each cluster different?
\end{quote}

\vspace{2em}

\begin{quote}
Do the results support your original question?
\end{quote}

\vspace{2em}

\begin{quote}
How would you describe each cluster in plain language?
\end{quote}

\vspace{2em}

\newpage

\section*{Step 10: Add Context from Other
Tables}\label{step-10-add-context-from-other-tables}
\addcontentsline{toc}{section}{Step 10: Add Context from Other Tables}

Join back to gtin or shopper\_info to see what's driving patterns.

Example: Most purchased subcategory by cluster.

\begin{Shaded}
\begin{Highlighting}[]
\NormalTok{your\_data }\SpecialCharTok{\%\textgreater{}\%}
  \FunctionTok{left\_join}\NormalTok{(cluster\_info, }\AttributeTok{by =} \StringTok{"id"}\NormalTok{) }\SpecialCharTok{\%\textgreater{}\%}
  \FunctionTok{group\_by}\NormalTok{(cluster, subcategory) }\SpecialCharTok{\%\textgreater{}\%}
  \FunctionTok{summarise}\NormalTok{(}\AttributeTok{count =} \FunctionTok{n}\NormalTok{()) }\SpecialCharTok{\%\textgreater{}\%}
  \FunctionTok{filter}\NormalTok{(count }\SpecialCharTok{==} \FunctionTok{max}\NormalTok{(count))  }\CommentTok{\# most common}
\end{Highlighting}
\end{Shaded}

\begin{center}\rule{0.5\linewidth}{0.5pt}\end{center}

\section*{Final Checklist}\label{final-checklist}
\addcontentsline{toc}{section}{Final Checklist}

\begin{itemize}
\item[$\square$]
  I have a clear and specific research question
\item[$\square$]
  I created a clean dataset at the right level of analysis
\item[$\square$]
  I chose 3--5 good variables for clustering
\item[$\square$]
  I transformed and scaled the data
\item[$\square$]
  I chose an appropriate number of clusters
\item[$\square$]
  I interpreted each cluster in context
\item[$\square$]
  I documented my decisions throughout
\end{itemize}




\end{document}
