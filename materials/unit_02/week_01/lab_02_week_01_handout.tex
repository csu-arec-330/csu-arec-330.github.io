% Options for packages loaded elsewhere
\PassOptionsToPackage{unicode}{hyperref}
\PassOptionsToPackage{hyphens}{url}
%
\documentclass[
  11pt,
]{article}
\usepackage{amsmath,amssymb}
\usepackage{iftex}
\ifPDFTeX
  \usepackage[T1]{fontenc}
  \usepackage[utf8]{inputenc}
  \usepackage{textcomp} % provide euro and other symbols
\else % if luatex or xetex
  \usepackage{unicode-math} % this also loads fontspec
  \defaultfontfeatures{Scale=MatchLowercase}
  \defaultfontfeatures[\rmfamily]{Ligatures=TeX,Scale=1}
\fi
\usepackage{lmodern}
\ifPDFTeX\else
  % xetex/luatex font selection
\fi
% Use upquote if available, for straight quotes in verbatim environments
\IfFileExists{upquote.sty}{\usepackage{upquote}}{}
\IfFileExists{microtype.sty}{% use microtype if available
  \usepackage[]{microtype}
  \UseMicrotypeSet[protrusion]{basicmath} % disable protrusion for tt fonts
}{}
\makeatletter
\@ifundefined{KOMAClassName}{% if non-KOMA class
  \IfFileExists{parskip.sty}{%
    \usepackage{parskip}
  }{% else
    \setlength{\parindent}{0pt}
    \setlength{\parskip}{6pt plus 2pt minus 1pt}}
}{% if KOMA class
  \KOMAoptions{parskip=half}}
\makeatother
\usepackage{xcolor}
\usepackage[margin=1in]{geometry}
\usepackage{longtable,booktabs,array}
\usepackage{calc} % for calculating minipage widths
% Correct order of tables after \paragraph or \subparagraph
\usepackage{etoolbox}
\makeatletter
\patchcmd\longtable{\par}{\if@noskipsec\mbox{}\fi\par}{}{}
\makeatother
% Allow footnotes in longtable head/foot
\IfFileExists{footnotehyper.sty}{\usepackage{footnotehyper}}{\usepackage{footnote}}
\makesavenoteenv{longtable}
\usepackage{graphicx}
\makeatletter
\def\maxwidth{\ifdim\Gin@nat@width>\linewidth\linewidth\else\Gin@nat@width\fi}
\def\maxheight{\ifdim\Gin@nat@height>\textheight\textheight\else\Gin@nat@height\fi}
\makeatother
% Scale images if necessary, so that they will not overflow the page
% margins by default, and it is still possible to overwrite the defaults
% using explicit options in \includegraphics[width, height, ...]{}
\setkeys{Gin}{width=\maxwidth,height=\maxheight,keepaspectratio}
% Set default figure placement to htbp
\makeatletter
\def\fps@figure{htbp}
\makeatother
\setlength{\emergencystretch}{3em} % prevent overfull lines
\providecommand{\tightlist}{%
  \setlength{\itemsep}{0pt}\setlength{\parskip}{0pt}}
\setcounter{secnumdepth}{-\maxdimen} % remove section numbering
\usepackage{setspace}
\setlength{\parindent}{0pt}
\ifLuaTeX
  \usepackage{selnolig}  % disable illegal ligatures
\fi
\usepackage{bookmark}
\IfFileExists{xurl.sty}{\usepackage{xurl}}{} % add URL line breaks if available
\urlstyle{same}
\hypersetup{
  pdftitle={Lab 02 Week 01 Worksheet},
  hidelinks,
  pdfcreator={LaTeX via pandoc}}

\title{Lab 02 Week 01 Worksheet}
\author{}
\date{\vspace{-2.5em}}

\begin{document}
\maketitle

\section{\texorpdfstring{\textbf{R Functions
Glossary}}{R Functions Glossary}}\label{r-functions-glossary}

This glossary provides an overview of key R functions used in
\textbf{Lab 09}, explaining their \textbf{purpose} and \textbf{general
use} in data processing and manipulation.

\subsection{Data Import \& Management}\label{data-import-management}

\begin{longtable}[]{@{}
  >{\raggedright\arraybackslash}p{(\columnwidth - 4\tabcolsep) * \real{0.2903}}
  >{\raggedright\arraybackslash}p{(\columnwidth - 4\tabcolsep) * \real{0.4194}}
  >{\raggedright\arraybackslash}p{(\columnwidth - 4\tabcolsep) * \real{0.2903}}@{}}
\toprule\noalign{}
\begin{minipage}[b]{\linewidth}\raggedright
Function
\end{minipage} & \begin{minipage}[b]{\linewidth}\raggedright
Description
\end{minipage} & \begin{minipage}[b]{\linewidth}\raggedright
Example
\end{minipage} \\
\midrule\noalign{}
\endhead
\bottomrule\noalign{}
\endlastfoot
\texttt{read\_csv()} & Reads a CSV file into a tibble &
\texttt{read\_csv("data.csv")} \\
\texttt{setwd()} / \texttt{getwd()} & Sets or gets the working directory
& \texttt{setwd("path/to/folder")} \\
\texttt{write\_csv()} & Writes a data frame to a CSV file &
\texttt{write\_csv(df,\ "output.csv")} \\
\end{longtable}

\subsection{Data Wrangling (dplyr)}\label{data-wrangling-dplyr}

\begin{longtable}[]{@{}
  >{\raggedright\arraybackslash}p{(\columnwidth - 4\tabcolsep) * \real{0.3125}}
  >{\raggedright\arraybackslash}p{(\columnwidth - 4\tabcolsep) * \real{0.4062}}
  >{\raggedright\arraybackslash}p{(\columnwidth - 4\tabcolsep) * \real{0.2812}}@{}}
\toprule\noalign{}
\begin{minipage}[b]{\linewidth}\raggedright
Function
\end{minipage} & \begin{minipage}[b]{\linewidth}\raggedright
Description
\end{minipage} & \begin{minipage}[b]{\linewidth}\raggedright
Example
\end{minipage} \\
\midrule\noalign{}
\endhead
\bottomrule\noalign{}
\endlastfoot
\texttt{filter()} & Filters rows based on condition(s) &
\texttt{filter(unit\_price\ \textgreater{}\ 0)} \\
\texttt{mutate()} & Adds or transforms variables &
\texttt{mutate(total\ =\ unit\_price\ *\ unit\_quantity)} \\
\texttt{arrange()} & Sorts rows by variables &
\texttt{arrange(shopper\_id)} \\
\texttt{drop\_na()} & Removes rows with missing values &
\texttt{drop\_na()} \\
\texttt{group\_by()} + \texttt{summarize()} & Groups data and summarizes
it &
\texttt{group\_by(shopper\_id)\ \%\textgreater{}\%\ summarize(avg\_items\ =\ mean(unit\_quantity))} \\
\texttt{distinct()} & Selects distinct rows &
\texttt{distinct(shopper\_id,\ store\_id)} \\
\texttt{select()} & Selects columns &
\texttt{select(total\_spent,\ avg\_items)} \\
\end{longtable}

\subsection{Joins}\label{joins}

\begin{longtable}[]{@{}
  >{\raggedright\arraybackslash}p{(\columnwidth - 4\tabcolsep) * \real{0.3125}}
  >{\raggedright\arraybackslash}p{(\columnwidth - 4\tabcolsep) * \real{0.4062}}
  >{\raggedright\arraybackslash}p{(\columnwidth - 4\tabcolsep) * \real{0.2812}}@{}}
\toprule\noalign{}
\begin{minipage}[b]{\linewidth}\raggedright
Function
\end{minipage} & \begin{minipage}[b]{\linewidth}\raggedright
Description
\end{minipage} & \begin{minipage}[b]{\linewidth}\raggedright
Example
\end{minipage} \\
\midrule\noalign{}
\endhead
\bottomrule\noalign{}
\endlastfoot
\texttt{inner\_join()} & Keeps only matching rows from both tables &
\texttt{inner\_join(df1,\ df2,\ by\ =\ "key")} \\
\texttt{left\_join()} & Keeps all rows from the left table &
\texttt{left\_join(df1,\ df2,\ by\ =\ "key")} \\
\texttt{right\_join()} & Keeps all rows from the right table &
\texttt{right\_join(df1,\ df2,\ by\ =\ "key")} \\
\texttt{full\_join()} & Combines all rows from both tables &
\texttt{full\_join(df1,\ df2,\ by\ =\ "key")} \\
\end{longtable}

\subsection{Summary Statistics}\label{summary-statistics}

\begin{longtable}[]{@{}
  >{\raggedright\arraybackslash}p{(\columnwidth - 4\tabcolsep) * \real{0.3125}}
  >{\raggedright\arraybackslash}p{(\columnwidth - 4\tabcolsep) * \real{0.4062}}
  >{\raggedright\arraybackslash}p{(\columnwidth - 4\tabcolsep) * \real{0.2812}}@{}}
\toprule\noalign{}
\begin{minipage}[b]{\linewidth}\raggedright
Function
\end{minipage} & \begin{minipage}[b]{\linewidth}\raggedright
Description
\end{minipage} & \begin{minipage}[b]{\linewidth}\raggedright
Example
\end{minipage} \\
\midrule\noalign{}
\endhead
\bottomrule\noalign{}
\endlastfoot
\texttt{datasummary\_skim()} & Summary of numeric or categorical data &
\texttt{datasummary\_skim(df,\ type\ =\ "numeric")} \\
\texttt{datasummary()} & Custom summary table &
\texttt{datasummary(var1\ +\ var2\ \textasciitilde{}\ Mean\ +\ SD,\ data\ =\ df)} \\
\end{longtable}

\subsection{Visualizations}\label{visualizations}

\begin{longtable}[]{@{}
  >{\raggedright\arraybackslash}p{(\columnwidth - 4\tabcolsep) * \real{0.3125}}
  >{\raggedright\arraybackslash}p{(\columnwidth - 4\tabcolsep) * \real{0.4062}}
  >{\raggedright\arraybackslash}p{(\columnwidth - 4\tabcolsep) * \real{0.2812}}@{}}
\toprule\noalign{}
\begin{minipage}[b]{\linewidth}\raggedright
Function
\end{minipage} & \begin{minipage}[b]{\linewidth}\raggedright
Description
\end{minipage} & \begin{minipage}[b]{\linewidth}\raggedright
Example
\end{minipage} \\
\midrule\noalign{}
\endhead
\bottomrule\noalign{}
\endlastfoot
\texttt{ggpairs()} & Pairwise scatter/density plots &
\texttt{ggpairs(df\ \%\textgreater{}\%\ select(x,\ y,\ z))} \\
\texttt{fviz\_nbclust()} & Plots to determine number of clusters &
\texttt{fviz\_nbclust(scaled\_data,\ kmeans,\ method\ =\ "wss")} \\
\end{longtable}

\subsection{Clustering}\label{clustering}

\begin{longtable}[]{@{}
  >{\raggedright\arraybackslash}p{(\columnwidth - 4\tabcolsep) * \real{0.3125}}
  >{\raggedright\arraybackslash}p{(\columnwidth - 4\tabcolsep) * \real{0.4062}}
  >{\raggedright\arraybackslash}p{(\columnwidth - 4\tabcolsep) * \real{0.2812}}@{}}
\toprule\noalign{}
\begin{minipage}[b]{\linewidth}\raggedright
Function
\end{minipage} & \begin{minipage}[b]{\linewidth}\raggedright
Description
\end{minipage} & \begin{minipage}[b]{\linewidth}\raggedright
Example
\end{minipage} \\
\midrule\noalign{}
\endhead
\bottomrule\noalign{}
\endlastfoot
\texttt{scale()} & Standardizes variables & \texttt{scale(df)} \\
\texttt{kmeans()} & Performs k-means clustering &
\texttt{kmeans(data,\ centers\ =\ 3,\ nstart\ =\ 25)} \\
\end{longtable}

\subsection{Other Tools \& Utilities}\label{other-tools-utilities}

\begin{longtable}[]{@{}
  >{\raggedright\arraybackslash}p{(\columnwidth - 4\tabcolsep) * \real{0.3125}}
  >{\raggedright\arraybackslash}p{(\columnwidth - 4\tabcolsep) * \real{0.4062}}
  >{\raggedright\arraybackslash}p{(\columnwidth - 4\tabcolsep) * \real{0.2812}}@{}}
\toprule\noalign{}
\begin{minipage}[b]{\linewidth}\raggedright
Function
\end{minipage} & \begin{minipage}[b]{\linewidth}\raggedright
Description
\end{minipage} & \begin{minipage}[b]{\linewidth}\raggedright
Example
\end{minipage} \\
\midrule\noalign{}
\endhead
\bottomrule\noalign{}
\endlastfoot
\texttt{length(unique())} & Counts distinct elements &
\texttt{length(unique(df\$shopper\_id))} \\
\texttt{quantile()} & Returns quantiles &
\texttt{quantile(df\$total\_spent,\ 0.999)} \\
\texttt{set.seed()} & Use this with \texttt{kmeans()} so you can
reproduce your results & \texttt{set.seed(123)} \\
\end{longtable}

\subsection{Helpful Tips}\label{helpful-tips}

\begin{itemize}
\tightlist
\item
  Use \texttt{left\_join()} when you want to \emph{keep all rows} from
  your base dataset.
\item
  Use \texttt{group\_by()} with \texttt{summarise()} to collapse and
  summarize grouped data.
\item
  Standardize your variables before clustering using \texttt{scale()}.
\item
  Use \texttt{fviz\_nbclust()} to help determine the best number of
  clusters.
\item
  Always inspect your summary statistics and visualize your data before
  running models.
\end{itemize}

\textbf{Remember}: Each step in data cleaning, joining, and clustering
depends on your research question. Document your decisions clearly!

\end{document}
