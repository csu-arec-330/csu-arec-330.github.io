% Options for packages loaded elsewhere
\PassOptionsToPackage{unicode}{hyperref}
\PassOptionsToPackage{hyphens}{url}
%
\documentclass[
  11pt,
]{article}
\usepackage{amsmath,amssymb}
\usepackage{iftex}
\ifPDFTeX
  \usepackage[T1]{fontenc}
  \usepackage[utf8]{inputenc}
  \usepackage{textcomp} % provide euro and other symbols
\else % if luatex or xetex
  \usepackage{unicode-math} % this also loads fontspec
  \defaultfontfeatures{Scale=MatchLowercase}
  \defaultfontfeatures[\rmfamily]{Ligatures=TeX,Scale=1}
\fi
\usepackage{lmodern}
\ifPDFTeX\else
  % xetex/luatex font selection
\fi
% Use upquote if available, for straight quotes in verbatim environments
\IfFileExists{upquote.sty}{\usepackage{upquote}}{}
\IfFileExists{microtype.sty}{% use microtype if available
  \usepackage[]{microtype}
  \UseMicrotypeSet[protrusion]{basicmath} % disable protrusion for tt fonts
}{}
\makeatletter
\@ifundefined{KOMAClassName}{% if non-KOMA class
  \IfFileExists{parskip.sty}{%
    \usepackage{parskip}
  }{% else
    \setlength{\parindent}{0pt}
    \setlength{\parskip}{6pt plus 2pt minus 1pt}}
}{% if KOMA class
  \KOMAoptions{parskip=half}}
\makeatother
\usepackage{xcolor}
\usepackage[margin=1in]{geometry}
\usepackage{longtable,booktabs,array}
\usepackage{calc} % for calculating minipage widths
% Correct order of tables after \paragraph or \subparagraph
\usepackage{etoolbox}
\makeatletter
\patchcmd\longtable{\par}{\if@noskipsec\mbox{}\fi\par}{}{}
\makeatother
% Allow footnotes in longtable head/foot
\IfFileExists{footnotehyper.sty}{\usepackage{footnotehyper}}{\usepackage{footnote}}
\makesavenoteenv{longtable}
\usepackage{graphicx}
\makeatletter
\def\maxwidth{\ifdim\Gin@nat@width>\linewidth\linewidth\else\Gin@nat@width\fi}
\def\maxheight{\ifdim\Gin@nat@height>\textheight\textheight\else\Gin@nat@height\fi}
\makeatother
% Scale images if necessary, so that they will not overflow the page
% margins by default, and it is still possible to overwrite the defaults
% using explicit options in \includegraphics[width, height, ...]{}
\setkeys{Gin}{width=\maxwidth,height=\maxheight,keepaspectratio}
% Set default figure placement to htbp
\makeatletter
\def\fps@figure{htbp}
\makeatother
\setlength{\emergencystretch}{3em} % prevent overfull lines
\providecommand{\tightlist}{%
  \setlength{\itemsep}{0pt}\setlength{\parskip}{0pt}}
\setcounter{secnumdepth}{-\maxdimen} % remove section numbering
\usepackage{setspace}
\setlength{\parindent}{0pt}
\ifLuaTeX
  \usepackage{selnolig}  % disable illegal ligatures
\fi
\usepackage{bookmark}
\IfFileExists{xurl.sty}{\usepackage{xurl}}{} % add URL line breaks if available
\urlstyle{same}
\hypersetup{
  pdftitle={Project 03 Handout},
  hidelinks,
  pdfcreator={LaTeX via pandoc}}

\title{Project 03 Handout}
\author{}
\date{\vspace{-2.5em}}

\begin{document}
\maketitle

\subsubsection{Objective}\label{objective}

Work with a simplified example from your Project 3 data to identify:

\begin{itemize}
\tightlist
\item
  The spatial unit of analysis\\
\item
  The type of spatial join required\\
\item
  Potential problems with the join
\end{itemize}

\begin{center}\rule{0.5\linewidth}{0.5pt}\end{center}

\subsubsection{Instructions}\label{instructions}

\begin{enumerate}
\def\labelenumi{\arabic{enumi}.}
\tightlist
\item
  \textbf{Get into pairs} (or groups of 3).
\item
  \textbf{Review the mini datasets} shown below.
\item
  As a group, answer the following questions:
\end{enumerate}

\textbf{Group Worksheet Questions}

\begin{enumerate}
\def\labelenumi{\arabic{enumi}.}
\tightlist
\item
  What is your outcome variable (\(y\))?
\item
  What is your explanatory variable (\(x\))?
\item
  What is the spatial unit of analysis in each dataset?
\item
  What type of spatial data is this? (point, polygon, raster)
\item
  What kind of spatial join is needed? (e.g., point-to-polygon,
  polygon-to-polygon)
\item
  What could go wrong with this join?
\end{enumerate}

\cleardoublepage

\subsubsection{Example Mini Datasets}\label{example-mini-datasets}

\textbf{Shopper Sample}

\begin{longtable}[]{@{}
  >{\raggedright\arraybackslash}p{(\columnwidth - 14\tabcolsep) * \real{0.1250}}
  >{\raggedright\arraybackslash}p{(\columnwidth - 14\tabcolsep) * \real{0.1250}}
  >{\raggedright\arraybackslash}p{(\columnwidth - 14\tabcolsep) * \real{0.1250}}
  >{\raggedright\arraybackslash}p{(\columnwidth - 14\tabcolsep) * \real{0.1250}}
  >{\raggedright\arraybackslash}p{(\columnwidth - 14\tabcolsep) * \real{0.1250}}
  >{\raggedright\arraybackslash}p{(\columnwidth - 14\tabcolsep) * \real{0.1250}}
  >{\raggedright\arraybackslash}p{(\columnwidth - 14\tabcolsep) * \real{0.1250}}
  >{\raggedright\arraybackslash}p{(\columnwidth - 14\tabcolsep) * \real{0.1250}}@{}}
\toprule\noalign{}
\begin{minipage}[b]{\linewidth}\raggedright
shopper\_id
\end{minipage} & \begin{minipage}[b]{\linewidth}\raggedright
store\_id
\end{minipage} & \begin{minipage}[b]{\linewidth}\raggedright
tr\_set\_id
\end{minipage} & \begin{minipage}[b]{\linewidth}\raggedright
gtin
\end{minipage} & \begin{minipage}[b]{\linewidth}\raggedright
price
\end{minipage} & \begin{minipage}[b]{\linewidth}\raggedright
quantity
\end{minipage} & \begin{minipage}[b]{\linewidth}\raggedright
purchase\_date
\end{minipage} & \begin{minipage}[b]{\linewidth}\raggedright
zip
\end{minipage} \\
\midrule\noalign{}
\endhead
\bottomrule\noalign{}
\endlastfoot
a1 & 101 & 10001 & 0001 & 2.99 & 1 & 2023-07-01 & 80021 \\
b2 & 102 & 10002 & 0002 & 1.50 & 2 & 2023-07-01 & 80022 \\
c3 & 103 & 10003 & 0003 & 3.25 & 1 & 2023-07-01 & 80023 \\
\end{longtable}

\textbf{Store Sample}

\begin{longtable}[]{@{}llllll@{}}
\toprule\noalign{}
store\_id & store\_name & zip\_code & latitude & longitude & \\
\midrule\noalign{}
\endhead
\bottomrule\noalign{}
\endlastfoot
101 & Store A & 80021 & 39.9205 & -105.092 & \\
102 & Store B & 80022 & 39.9206 & -105.093 & \\
103 & Store C & 80023 & 39.9207 & -105.094 & \\
\end{longtable}

\textbf{Census Sample}

\begin{longtable}[]{@{}llll@{}}
\toprule\noalign{}
zip\_code & population & median\_income & percent\_college\_degree \\
\midrule\noalign{}
\endhead
\bottomrule\noalign{}
\endlastfoot
80021 & 25,000 & 68,000 & 0.45 \\
80022 & 18,000 & 59,000 & 0.32 \\
80023 & 22,000 & 63,000 & 0.38 \\
\end{longtable}

\textbf{Weather Sample}

\begin{longtable}[]{@{}llll@{}}
\toprule\noalign{}
zip\_code & date & avg\_temp & precip \\
\midrule\noalign{}
\endhead
\bottomrule\noalign{}
\endlastfoot
80021 & 2023-07-01 & 85.2 & 0.10 \\
80022 & 2023-07-01 & 87.1 & 0.00 \\
80023 & 2023-07-01 & 86.4 & 0.05 \\
\end{longtable}

\end{document}
