% Options for packages loaded elsewhere
\PassOptionsToPackage{unicode}{hyperref}
\PassOptionsToPackage{hyphens}{url}
%
\documentclass[
  11pt,
]{article}
\usepackage{amsmath,amssymb}
\usepackage{iftex}
\ifPDFTeX
  \usepackage[T1]{fontenc}
  \usepackage[utf8]{inputenc}
  \usepackage{textcomp} % provide euro and other symbols
\else % if luatex or xetex
  \usepackage{unicode-math} % this also loads fontspec
  \defaultfontfeatures{Scale=MatchLowercase}
  \defaultfontfeatures[\rmfamily]{Ligatures=TeX,Scale=1}
\fi
\usepackage{lmodern}
\ifPDFTeX\else
  % xetex/luatex font selection
\fi
% Use upquote if available, for straight quotes in verbatim environments
\IfFileExists{upquote.sty}{\usepackage{upquote}}{}
\IfFileExists{microtype.sty}{% use microtype if available
  \usepackage[]{microtype}
  \UseMicrotypeSet[protrusion]{basicmath} % disable protrusion for tt fonts
}{}
\makeatletter
\@ifundefined{KOMAClassName}{% if non-KOMA class
  \IfFileExists{parskip.sty}{%
    \usepackage{parskip}
  }{% else
    \setlength{\parindent}{0pt}
    \setlength{\parskip}{6pt plus 2pt minus 1pt}}
}{% if KOMA class
  \KOMAoptions{parskip=half}}
\makeatother
\usepackage{xcolor}
\usepackage[margin=1in]{geometry}
\usepackage{color}
\usepackage{fancyvrb}
\newcommand{\VerbBar}{|}
\newcommand{\VERB}{\Verb[commandchars=\\\{\}]}
\DefineVerbatimEnvironment{Highlighting}{Verbatim}{commandchars=\\\{\}}
% Add ',fontsize=\small' for more characters per line
\usepackage{framed}
\definecolor{shadecolor}{RGB}{248,248,248}
\newenvironment{Shaded}{\begin{snugshade}}{\end{snugshade}}
\newcommand{\AlertTok}[1]{\textcolor[rgb]{0.94,0.16,0.16}{#1}}
\newcommand{\AnnotationTok}[1]{\textcolor[rgb]{0.56,0.35,0.01}{\textbf{\textit{#1}}}}
\newcommand{\AttributeTok}[1]{\textcolor[rgb]{0.13,0.29,0.53}{#1}}
\newcommand{\BaseNTok}[1]{\textcolor[rgb]{0.00,0.00,0.81}{#1}}
\newcommand{\BuiltInTok}[1]{#1}
\newcommand{\CharTok}[1]{\textcolor[rgb]{0.31,0.60,0.02}{#1}}
\newcommand{\CommentTok}[1]{\textcolor[rgb]{0.56,0.35,0.01}{\textit{#1}}}
\newcommand{\CommentVarTok}[1]{\textcolor[rgb]{0.56,0.35,0.01}{\textbf{\textit{#1}}}}
\newcommand{\ConstantTok}[1]{\textcolor[rgb]{0.56,0.35,0.01}{#1}}
\newcommand{\ControlFlowTok}[1]{\textcolor[rgb]{0.13,0.29,0.53}{\textbf{#1}}}
\newcommand{\DataTypeTok}[1]{\textcolor[rgb]{0.13,0.29,0.53}{#1}}
\newcommand{\DecValTok}[1]{\textcolor[rgb]{0.00,0.00,0.81}{#1}}
\newcommand{\DocumentationTok}[1]{\textcolor[rgb]{0.56,0.35,0.01}{\textbf{\textit{#1}}}}
\newcommand{\ErrorTok}[1]{\textcolor[rgb]{0.64,0.00,0.00}{\textbf{#1}}}
\newcommand{\ExtensionTok}[1]{#1}
\newcommand{\FloatTok}[1]{\textcolor[rgb]{0.00,0.00,0.81}{#1}}
\newcommand{\FunctionTok}[1]{\textcolor[rgb]{0.13,0.29,0.53}{\textbf{#1}}}
\newcommand{\ImportTok}[1]{#1}
\newcommand{\InformationTok}[1]{\textcolor[rgb]{0.56,0.35,0.01}{\textbf{\textit{#1}}}}
\newcommand{\KeywordTok}[1]{\textcolor[rgb]{0.13,0.29,0.53}{\textbf{#1}}}
\newcommand{\NormalTok}[1]{#1}
\newcommand{\OperatorTok}[1]{\textcolor[rgb]{0.81,0.36,0.00}{\textbf{#1}}}
\newcommand{\OtherTok}[1]{\textcolor[rgb]{0.56,0.35,0.01}{#1}}
\newcommand{\PreprocessorTok}[1]{\textcolor[rgb]{0.56,0.35,0.01}{\textit{#1}}}
\newcommand{\RegionMarkerTok}[1]{#1}
\newcommand{\SpecialCharTok}[1]{\textcolor[rgb]{0.81,0.36,0.00}{\textbf{#1}}}
\newcommand{\SpecialStringTok}[1]{\textcolor[rgb]{0.31,0.60,0.02}{#1}}
\newcommand{\StringTok}[1]{\textcolor[rgb]{0.31,0.60,0.02}{#1}}
\newcommand{\VariableTok}[1]{\textcolor[rgb]{0.00,0.00,0.00}{#1}}
\newcommand{\VerbatimStringTok}[1]{\textcolor[rgb]{0.31,0.60,0.02}{#1}}
\newcommand{\WarningTok}[1]{\textcolor[rgb]{0.56,0.35,0.01}{\textbf{\textit{#1}}}}
\usepackage{graphicx}
\makeatletter
\def\maxwidth{\ifdim\Gin@nat@width>\linewidth\linewidth\else\Gin@nat@width\fi}
\def\maxheight{\ifdim\Gin@nat@height>\textheight\textheight\else\Gin@nat@height\fi}
\makeatother
% Scale images if necessary, so that they will not overflow the page
% margins by default, and it is still possible to overwrite the defaults
% using explicit options in \includegraphics[width, height, ...]{}
\setkeys{Gin}{width=\maxwidth,height=\maxheight,keepaspectratio}
% Set default figure placement to htbp
\makeatletter
\def\fps@figure{htbp}
\makeatother
\setlength{\emergencystretch}{3em} % prevent overfull lines
\providecommand{\tightlist}{%
  \setlength{\itemsep}{0pt}\setlength{\parskip}{0pt}}
\setcounter{secnumdepth}{-\maxdimen} % remove section numbering
\usepackage{setspace}
\setlength{\parindent}{0pt}
\ifLuaTeX
  \usepackage{selnolig}  % disable illegal ligatures
\fi
\usepackage{bookmark}
\IfFileExists{xurl.sty}{\usepackage{xurl}}{} % add URL line breaks if available
\urlstyle{same}
\hypersetup{
  pdftitle={Lab 01 Week 05 Worksheet},
  hidelinks,
  pdfcreator={LaTeX via pandoc}}

\title{Lab 01 Week 05 Worksheet}
\author{}
\date{\vspace{-2.5em}}

\begin{document}
\maketitle

\section{\texorpdfstring{\textbf{R Functions
Glossary}}{R Functions Glossary}}\label{r-functions-glossary}

This glossary provides an overview of key R functions used in
\textbf{Week 5 Lab}, explaining their \textbf{purpose} and
\textbf{general use} in handling time series data.

\begin{center}\rule{0.5\linewidth}{0.5pt}\end{center}

\subsection{\texorpdfstring{\textbf{1. Downloading
Data}}{1. Downloading Data}}\label{downloading-data}

\subsubsection{\texorpdfstring{\textbf{\texttt{tq\_get(symbol,\ get\ =\ "data\_source",\ from\ =\ "YYYY-MM-DD",\ to\ =\ "YYYY-MM-DD")}}}{tq\_get(symbol, get = "data\_source", from = "YYYY-MM-DD", to = "YYYY-MM-DD")}}\label{tq_getsymbol-get-data_source-from-yyyy-mm-dd-to-yyyy-mm-dd}

\textbf{Purpose:} Retrieves time series data from FRED, Yahoo Finance,
or other sources.\\
\textbf{Example:}

\begin{Shaded}
\begin{Highlighting}[]
\NormalTok{data }\OtherTok{\textless{}{-}} \FunctionTok{tq\_get}\NormalTok{(}\StringTok{"AAPL"}\NormalTok{, }\AttributeTok{get =} \StringTok{"stock.prices"}\NormalTok{, }\AttributeTok{from =} \StringTok{"2020{-}01{-}01"}\NormalTok{, }\AttributeTok{to =} \StringTok{"2024{-}02{-}28"}\NormalTok{)}
\end{Highlighting}
\end{Shaded}

\begin{center}\rule{0.5\linewidth}{0.5pt}\end{center}

\subsection{\texorpdfstring{\textbf{2. Handling
Dates}}{2. Handling Dates}}\label{handling-dates}

\subsubsection{\texorpdfstring{\textbf{\texttt{as.numeric(Sys.Date())}}}{as.numeric(Sys.Date())}}\label{as.numericsys.date}

\textbf{Purpose:} Converts the current date into a numeric value based
on the R epoch (January 1, 1970).\\
\textbf{Example:}

\begin{Shaded}
\begin{Highlighting}[]
\FunctionTok{as.numeric}\NormalTok{(}\FunctionTok{Sys.Date}\NormalTok{())}
\end{Highlighting}
\end{Shaded}

\subsubsection{\texorpdfstring{\textbf{\texttt{ymd("YYYY-MM-DD")}}}{ymd("YYYY-MM-DD")}}\label{ymdyyyy-mm-dd}

\textbf{Purpose:} Converts a character string into a date format.\\
\textbf{Example:}

\begin{Shaded}
\begin{Highlighting}[]
\NormalTok{date }\OtherTok{\textless{}{-}} \FunctionTok{ymd}\NormalTok{(}\StringTok{"2024{-}02{-}28"}\NormalTok{)}
\end{Highlighting}
\end{Shaded}

\begin{center}\rule{0.5\linewidth}{0.5pt}\end{center}

\subsection{\texorpdfstring{\textbf{3. Modifying
Data}}{3. Modifying Data}}\label{modifying-data}

\subsubsection{\texorpdfstring{\textbf{\texttt{mutate(dataframe,\ new\_column\ =\ operation)}}}{mutate(dataframe, new\_column = operation)}}\label{mutatedataframe-new_column-operation}

\textbf{Purpose:} Adds or transforms columns in a dataframe.\\
\textbf{Example:}

\begin{Shaded}
\begin{Highlighting}[]
\NormalTok{data }\OtherTok{\textless{}{-}} \FunctionTok{mutate}\NormalTok{(data, }\AttributeTok{price\_change =}\NormalTok{ price }\SpecialCharTok{{-}} \FunctionTok{lag}\NormalTok{(price))}
\end{Highlighting}
\end{Shaded}

\subsubsection{\texorpdfstring{\textbf{\texttt{case\_when()}}}{case\_when()}}\label{case_when}

\textbf{Purpose:} Recodes values based on conditions.\\
\textbf{Example:}

\begin{Shaded}
\begin{Highlighting}[]
\NormalTok{data }\OtherTok{\textless{}{-}} \FunctionTok{mutate}\NormalTok{(data, }\AttributeTok{category =} \FunctionTok{case\_when}\NormalTok{(price }\SpecialCharTok{\textgreater{}} \DecValTok{100} \SpecialCharTok{\textasciitilde{}} \StringTok{"High"}\NormalTok{, }\ConstantTok{TRUE} \SpecialCharTok{\textasciitilde{}} \StringTok{"Low"}\NormalTok{))}
\end{Highlighting}
\end{Shaded}

\begin{center}\rule{0.5\linewidth}{0.5pt}\end{center}

\subsection{\texorpdfstring{\textbf{4. Exporting
Data}}{4. Exporting Data}}\label{exporting-data}

\subsubsection{\texorpdfstring{\textbf{\texttt{write\_csv(dataframe,\ "filename.csv")}}}{write\_csv(dataframe, "filename.csv")}}\label{write_csvdataframe-filename.csv}

\textbf{Purpose:} Saves a dataframe as a CSV file for use in Tableau.\\
\textbf{Example:}

\begin{Shaded}
\begin{Highlighting}[]
\FunctionTok{write\_csv}\NormalTok{(data, }\StringTok{"time\_series\_data.csv"}\NormalTok{)}
\end{Highlighting}
\end{Shaded}

\begin{center}\rule{0.5\linewidth}{0.5pt}\end{center}

\subsection{\texorpdfstring{\textbf{Using This
Glossary}}{Using This Glossary}}\label{using-this-glossary}

\begin{itemize}
\tightlist
\item
  Reference this list while working through Week 5 Lab.
\item
  Experiment with each function in R to understand how it works.
\item
  Use piping (\texttt{\%\textgreater{}\%}) to combine multiple functions
  and streamline analysis.
\end{itemize}

\end{document}
