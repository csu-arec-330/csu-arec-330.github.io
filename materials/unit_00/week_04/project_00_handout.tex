% Options for packages loaded elsewhere
\PassOptionsToPackage{unicode}{hyperref}
\PassOptionsToPackage{hyphens}{url}
%
\documentclass[
  11pt,
]{article}
\usepackage{amsmath,amssymb}
\usepackage{iftex}
\ifPDFTeX
  \usepackage[T1]{fontenc}
  \usepackage[utf8]{inputenc}
  \usepackage{textcomp} % provide euro and other symbols
\else % if luatex or xetex
  \usepackage{unicode-math} % this also loads fontspec
  \defaultfontfeatures{Scale=MatchLowercase}
  \defaultfontfeatures[\rmfamily]{Ligatures=TeX,Scale=1}
\fi
\usepackage{lmodern}
\ifPDFTeX\else
  % xetex/luatex font selection
\fi
% Use upquote if available, for straight quotes in verbatim environments
\IfFileExists{upquote.sty}{\usepackage{upquote}}{}
\IfFileExists{microtype.sty}{% use microtype if available
  \usepackage[]{microtype}
  \UseMicrotypeSet[protrusion]{basicmath} % disable protrusion for tt fonts
}{}
\makeatletter
\@ifundefined{KOMAClassName}{% if non-KOMA class
  \IfFileExists{parskip.sty}{%
    \usepackage{parskip}
  }{% else
    \setlength{\parindent}{0pt}
    \setlength{\parskip}{6pt plus 2pt minus 1pt}}
}{% if KOMA class
  \KOMAoptions{parskip=half}}
\makeatother
\usepackage{xcolor}
\usepackage[margin=1in]{geometry}
\usepackage{graphicx}
\makeatletter
\def\maxwidth{\ifdim\Gin@nat@width>\linewidth\linewidth\else\Gin@nat@width\fi}
\def\maxheight{\ifdim\Gin@nat@height>\textheight\textheight\else\Gin@nat@height\fi}
\makeatother
% Scale images if necessary, so that they will not overflow the page
% margins by default, and it is still possible to overwrite the defaults
% using explicit options in \includegraphics[width, height, ...]{}
\setkeys{Gin}{width=\maxwidth,height=\maxheight,keepaspectratio}
% Set default figure placement to htbp
\makeatletter
\def\fps@figure{htbp}
\makeatother
\setlength{\emergencystretch}{3em} % prevent overfull lines
\providecommand{\tightlist}{%
  \setlength{\itemsep}{0pt}\setlength{\parskip}{0pt}}
\setcounter{secnumdepth}{-\maxdimen} % remove section numbering
\usepackage{setspace}
\setlength{\parindent}{0pt}
\ifLuaTeX
  \usepackage{selnolig}  % disable illegal ligatures
\fi
\usepackage{bookmark}
\IfFileExists{xurl.sty}{\usepackage{xurl}}{} % add URL line breaks if available
\urlstyle{same}
\hypersetup{
  pdftitle={Project 00 (Problem Set 4) Handout},
  hidelinks,
  pdfcreator={LaTeX via pandoc}}

\title{Project 00 (Problem Set 4) Handout}
\author{}
\date{\vspace{-2.5em}}

\begin{document}
\maketitle

\section{Project Overview}\label{project-overview}

In this project, you will apply the D\textsuperscript{3}M process to
analyze supermarket sales data and develop actionable insights. Your
goal is to choose a business question, analyze the dataset, create
compelling visualizations, and present your findings in a structured
format.

\section{Project Steps}\label{project-steps}

\subsection{1. Choose a Question}\label{choose-a-question}

Select one of the questions below or propose your own (must be
answerable with the available dataset).

\subsubsection{Example Questions:}\label{example-questions}

\textbf{Sales \& Revenue Analysis}

\begin{itemize}
\tightlist
\item
  How do invoice values and branch contributions to revenue vary over
  time?
\end{itemize}

\textbf{Customer Behavior}

\begin{itemize}
\tightlist
\item
  What are the most popular product lines, and how frequently do
  customers purchase them?
\end{itemize}

\textbf{Customer Satisfaction}

\begin{itemize}
\tightlist
\item
  How do factors such as product line, time of purchase, payment method,
  and store location impact customer ratings?
\end{itemize}

\textbf{Payment Methods}

\begin{itemize}
\tightlist
\item
  Which channels of payment methods contribute most to sales, and how
  should payment methods be adjusted?
\end{itemize}

\subsubsection{2. Develop a Hypothesis}\label{develop-a-hypothesis}

What do you expect to find? Why?

Example: ``We hypothesize that electronic accessories have the highest
total sales, but health and beauty products receive the highest customer
ratings.''

\subsubsection{3. Explore \& Prepare Data}\label{explore-prepare-data}

\begin{itemize}
\item
  Load the \texttt{supermarket\_sales.csv} dataset in R.
\item
  Identify relevant variables that relate to your question.
\item
  Clean and prepare the data for analysis (e.g., filter, group,
  aggregate).
\end{itemize}

\subsubsection{4. Analyze the Data}\label{analyze-the-data}

What patterns and trends emerge?

\begin{itemize}
\item
  Use appropriate statistical measures (averages, totals, percentages)
  to support your findings.
\item
  Perform comparisons across different factors (e.g., time, product
  line, location).
\end{itemize}

\subsubsection{5. Visualize the Data}\label{visualize-the-data}

Tell a clear story with the visuals. Select the most effective visuals
to communicate your insights (e.g., bar charts, line graphs, heat maps).

Follow the six guidelines for effective visualizations covered in class:

\begin{enumerate}
\def\labelenumi{\arabic{enumi}.}
\item
  Identify your objective
\item
  Show the data
\item
  Reduce the clutter
\end{enumerate}

\begin{itemize}
\tightlist
\item
  Label axes and use appropriate scales
\end{itemize}

\begin{enumerate}
\def\labelenumi{\arabic{enumi}.}
\setcounter{enumi}{3}
\tightlist
\item
  Integrate the graphics and text
\end{enumerate}

\begin{itemize}
\item
  Use active titles (e.g., ``Weekend Sales Surpass Weekday Sales''
  instead of ``Sales by Day'')
\item
  Use whitespace effectively
\end{itemize}

\begin{enumerate}
\def\labelenumi{\arabic{enumi}.}
\setcounter{enumi}{4}
\item
  Avoid the spaghetti chart
\item
  Start with gray
\end{enumerate}

\subsubsection{6. Interpret Your Results}\label{interpret-your-results}

What do your findings reveal?

\begin{itemize}
\item
  Do the results support or contradict your hypothesis?
\item
  What are the limitations of your analysis?
\end{itemize}

\subsubsection{7. Make Recommendations}\label{make-recommendations}

Based on your findings, what actionable steps would you suggest?

Example: ``Since members purchase more health and beauty products on
weekends, the store should offer weekend promotions for this category.''

\end{document}
