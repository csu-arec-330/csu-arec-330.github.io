% Options for packages loaded elsewhere
\PassOptionsToPackage{unicode}{hyperref}
\PassOptionsToPackage{hyphens}{url}
%
\documentclass[
  11pt,
]{article}
\usepackage{amsmath,amssymb}
\usepackage{iftex}
\ifPDFTeX
  \usepackage[T1]{fontenc}
  \usepackage[utf8]{inputenc}
  \usepackage{textcomp} % provide euro and other symbols
\else % if luatex or xetex
  \usepackage{unicode-math} % this also loads fontspec
  \defaultfontfeatures{Scale=MatchLowercase}
  \defaultfontfeatures[\rmfamily]{Ligatures=TeX,Scale=1}
\fi
\usepackage{lmodern}
\ifPDFTeX\else
  % xetex/luatex font selection
\fi
% Use upquote if available, for straight quotes in verbatim environments
\IfFileExists{upquote.sty}{\usepackage{upquote}}{}
\IfFileExists{microtype.sty}{% use microtype if available
  \usepackage[]{microtype}
  \UseMicrotypeSet[protrusion]{basicmath} % disable protrusion for tt fonts
}{}
\makeatletter
\@ifundefined{KOMAClassName}{% if non-KOMA class
  \IfFileExists{parskip.sty}{%
    \usepackage{parskip}
  }{% else
    \setlength{\parindent}{0pt}
    \setlength{\parskip}{6pt plus 2pt minus 1pt}}
}{% if KOMA class
  \KOMAoptions{parskip=half}}
\makeatother
\usepackage{xcolor}
\usepackage[margin=1in]{geometry}
\usepackage{color}
\usepackage{fancyvrb}
\newcommand{\VerbBar}{|}
\newcommand{\VERB}{\Verb[commandchars=\\\{\}]}
\DefineVerbatimEnvironment{Highlighting}{Verbatim}{commandchars=\\\{\}}
% Add ',fontsize=\small' for more characters per line
\usepackage{framed}
\definecolor{shadecolor}{RGB}{248,248,248}
\newenvironment{Shaded}{\begin{snugshade}}{\end{snugshade}}
\newcommand{\AlertTok}[1]{\textcolor[rgb]{0.94,0.16,0.16}{#1}}
\newcommand{\AnnotationTok}[1]{\textcolor[rgb]{0.56,0.35,0.01}{\textbf{\textit{#1}}}}
\newcommand{\AttributeTok}[1]{\textcolor[rgb]{0.13,0.29,0.53}{#1}}
\newcommand{\BaseNTok}[1]{\textcolor[rgb]{0.00,0.00,0.81}{#1}}
\newcommand{\BuiltInTok}[1]{#1}
\newcommand{\CharTok}[1]{\textcolor[rgb]{0.31,0.60,0.02}{#1}}
\newcommand{\CommentTok}[1]{\textcolor[rgb]{0.56,0.35,0.01}{\textit{#1}}}
\newcommand{\CommentVarTok}[1]{\textcolor[rgb]{0.56,0.35,0.01}{\textbf{\textit{#1}}}}
\newcommand{\ConstantTok}[1]{\textcolor[rgb]{0.56,0.35,0.01}{#1}}
\newcommand{\ControlFlowTok}[1]{\textcolor[rgb]{0.13,0.29,0.53}{\textbf{#1}}}
\newcommand{\DataTypeTok}[1]{\textcolor[rgb]{0.13,0.29,0.53}{#1}}
\newcommand{\DecValTok}[1]{\textcolor[rgb]{0.00,0.00,0.81}{#1}}
\newcommand{\DocumentationTok}[1]{\textcolor[rgb]{0.56,0.35,0.01}{\textbf{\textit{#1}}}}
\newcommand{\ErrorTok}[1]{\textcolor[rgb]{0.64,0.00,0.00}{\textbf{#1}}}
\newcommand{\ExtensionTok}[1]{#1}
\newcommand{\FloatTok}[1]{\textcolor[rgb]{0.00,0.00,0.81}{#1}}
\newcommand{\FunctionTok}[1]{\textcolor[rgb]{0.13,0.29,0.53}{\textbf{#1}}}
\newcommand{\ImportTok}[1]{#1}
\newcommand{\InformationTok}[1]{\textcolor[rgb]{0.56,0.35,0.01}{\textbf{\textit{#1}}}}
\newcommand{\KeywordTok}[1]{\textcolor[rgb]{0.13,0.29,0.53}{\textbf{#1}}}
\newcommand{\NormalTok}[1]{#1}
\newcommand{\OperatorTok}[1]{\textcolor[rgb]{0.81,0.36,0.00}{\textbf{#1}}}
\newcommand{\OtherTok}[1]{\textcolor[rgb]{0.56,0.35,0.01}{#1}}
\newcommand{\PreprocessorTok}[1]{\textcolor[rgb]{0.56,0.35,0.01}{\textit{#1}}}
\newcommand{\RegionMarkerTok}[1]{#1}
\newcommand{\SpecialCharTok}[1]{\textcolor[rgb]{0.81,0.36,0.00}{\textbf{#1}}}
\newcommand{\SpecialStringTok}[1]{\textcolor[rgb]{0.31,0.60,0.02}{#1}}
\newcommand{\StringTok}[1]{\textcolor[rgb]{0.31,0.60,0.02}{#1}}
\newcommand{\VariableTok}[1]{\textcolor[rgb]{0.00,0.00,0.00}{#1}}
\newcommand{\VerbatimStringTok}[1]{\textcolor[rgb]{0.31,0.60,0.02}{#1}}
\newcommand{\WarningTok}[1]{\textcolor[rgb]{0.56,0.35,0.01}{\textbf{\textit{#1}}}}
\usepackage{graphicx}
\makeatletter
\def\maxwidth{\ifdim\Gin@nat@width>\linewidth\linewidth\else\Gin@nat@width\fi}
\def\maxheight{\ifdim\Gin@nat@height>\textheight\textheight\else\Gin@nat@height\fi}
\makeatother
% Scale images if necessary, so that they will not overflow the page
% margins by default, and it is still possible to overwrite the defaults
% using explicit options in \includegraphics[width, height, ...]{}
\setkeys{Gin}{width=\maxwidth,height=\maxheight,keepaspectratio}
% Set default figure placement to htbp
\makeatletter
\def\fps@figure{htbp}
\makeatother
\setlength{\emergencystretch}{3em} % prevent overfull lines
\providecommand{\tightlist}{%
  \setlength{\itemsep}{0pt}\setlength{\parskip}{0pt}}
\setcounter{secnumdepth}{-\maxdimen} % remove section numbering
\usepackage{setspace}
\setlength{\parindent}{0pt}
\ifLuaTeX
  \usepackage{selnolig}  % disable illegal ligatures
\fi
\usepackage{bookmark}
\IfFileExists{xurl.sty}{\usepackage{xurl}}{} % add URL line breaks if available
\urlstyle{same}
\hypersetup{
  pdftitle={Lab 00 Week 03 Worksheet},
  hidelinks,
  pdfcreator={LaTeX via pandoc}}

\title{Lab 00 Week 03 Worksheet}
\author{}
\date{\vspace{-2.5em}}

\begin{document}
\maketitle

\section{\texorpdfstring{\textbf{R Functions
Glossary}}{R Functions Glossary}}\label{r-functions-glossary}

This glossary provides an overview of key R functions used in
\textbf{Lab 03}, explaining their \textbf{purpose} and \textbf{general
use} in data processing and manipulation.

\begin{center}\rule{0.5\linewidth}{0.5pt}\end{center}

\subsection{\texorpdfstring{\textbf{1. Working with Directories and
Files}}{1. Working with Directories and Files}}\label{working-with-directories-and-files}

\subsubsection{\texorpdfstring{\textbf{\texttt{setwd("path")}}}{setwd("path")}}\label{setwdpath}

\textbf{Purpose:} Sets the working directory so that R knows where to
look for files.\\
\textbf{Example:}
\texttt{setwd("\textasciitilde{}/Documents/arec330/lab\_03")}

\subsubsection{\texorpdfstring{\textbf{\texttt{getwd()}}}{getwd()}}\label{getwd}

\textbf{Purpose:} Returns the current working directory.\\
\textbf{Example:} \texttt{getwd()}

\subsubsection{\texorpdfstring{\textbf{\texttt{dir()}}}{dir()}}\label{dir}

\textbf{Purpose:} Lists all files in the working directory.\\
\textbf{Example:} \texttt{dir()}

\begin{center}\rule{0.5\linewidth}{0.5pt}\end{center}

\subsection{\texorpdfstring{\textbf{2. Installing and Loading
Packages}}{2. Installing and Loading Packages}}\label{installing-and-loading-packages}

\subsubsection{\texorpdfstring{\textbf{\texttt{install.packages("package\_name")}}}{install.packages("package\_name")}}\label{install.packagespackage_name}

\textbf{Purpose:} Installs a package in R (only needs to be done once
per computer).\\
\textbf{Example:} \texttt{install.packages("dplyr")}

\subsubsection{\texorpdfstring{\textbf{\texttt{library(package\_name)}}}{library(package\_name)}}\label{librarypackage_name}

\textbf{Purpose:} Loads an installed package into the current session.\\
\textbf{Example:} \texttt{library(dplyr)}

\begin{center}\rule{0.5\linewidth}{0.5pt}\end{center}

\subsection{\texorpdfstring{\textbf{3. Reading and Viewing
Data}}{3. Reading and Viewing Data}}\label{reading-and-viewing-data}

\subsubsection{\texorpdfstring{\textbf{\texttt{read\_csv("file.csv")}}}{read\_csv("file.csv")}}\label{read_csvfile.csv}

\textbf{Purpose:} Reads a CSV file into R as a \textbf{tibble} (modern
dataframe).\\
\textbf{Example:}
\texttt{df\ \textless{}-\ read\_csv("supermarket\_sales.csv")}

\subsubsection{\texorpdfstring{\textbf{\texttt{View(df)}}}{View(df)}}\label{viewdf}

\textbf{Purpose:} Opens a spreadsheet-like viewer for a dataframe in
RStudio.\\
\textbf{Example:} \texttt{View(df)}

\subsubsection{\texorpdfstring{\textbf{\texttt{glimpse(df)}}}{glimpse(df)}}\label{glimpsedf}

\textbf{Purpose:} Provides a compact summary of a dataframe's
structure.\\
\textbf{Example:} \texttt{glimpse(df)}

\subsubsection{\texorpdfstring{\textbf{\texttt{str(df)}}}{str(df)}}\label{strdf}

\textbf{Purpose:} Shows the structure of an object, including variable
types.\\
\textbf{Example:} \texttt{str(df)}

\begin{center}\rule{0.5\linewidth}{0.5pt}\end{center}

\subsection{\texorpdfstring{\textbf{4. Cleaning and Transforming
Data}}{4. Cleaning and Transforming Data}}\label{cleaning-and-transforming-data}

\subsubsection{\texorpdfstring{\textbf{\texttt{clean\_names(df)}}}{clean\_names(df)}}\label{clean_namesdf}

\textbf{Purpose:} Cleans column names (e.g., converts spaces to
underscores).\\
\textbf{Example:} \texttt{df\ \textless{}-\ clean\_names(df)}

\subsubsection{\texorpdfstring{\textbf{\texttt{mutate(df,\ new\_column\ =\ operation)}}}{mutate(df, new\_column = operation)}}\label{mutatedf-new_column-operation}

\textbf{Purpose:} Creates or modifies columns in a dataframe.\\
\textbf{Example:}
\texttt{df\ \textless{}-\ mutate(df,\ total\_cost\ =\ unit\_price\ *\ quantity)}

\subsubsection{\texorpdfstring{\textbf{\texttt{rename(df,\ new\_name\ =\ old\_name)}}}{rename(df, new\_name = old\_name)}}\label{renamedf-new_name-old_name}

\textbf{Purpose:} Renames columns in a dataframe.\\
\textbf{Example:}
\texttt{df\ \textless{}-\ rename(df,\ market\ =\ city)}

\begin{center}\rule{0.5\linewidth}{0.5pt}\end{center}

\subsection{\texorpdfstring{\textbf{5. Filtering and Selecting
Data}}{5. Filtering and Selecting Data}}\label{filtering-and-selecting-data}

\subsubsection{\texorpdfstring{\textbf{\texttt{filter(df,\ condition)}}}{filter(df, condition)}}\label{filterdf-condition}

\textbf{Purpose:} Keeps only rows that meet a condition.\\
\textbf{Example:}
\texttt{df\_yangon\ \textless{}-\ filter(df,\ city\ ==\ "Yangon")}

\subsubsection{\texorpdfstring{\textbf{\texttt{select(df,\ column1,\ column2)}}}{select(df, column1, column2)}}\label{selectdf-column1-column2}

\textbf{Purpose:} Selects specific columns from a dataframe.\\
\textbf{Example:}
\texttt{df\_subset\ \textless{}-\ select(df,\ city,\ total,\ product\_line)}

\begin{center}\rule{0.5\linewidth}{0.5pt}\end{center}

\subsection{\texorpdfstring{\textbf{6. Sorting and Summarizing
Data}}{6. Sorting and Summarizing Data}}\label{sorting-and-summarizing-data}

\subsubsection{\texorpdfstring{\textbf{\texttt{arrange(df,\ column)}}}{arrange(df, column)}}\label{arrangedf-column}

\textbf{Purpose:} Sorts rows by a column in ascending order (default).\\
\textbf{Example:} \texttt{df\_sorted\ \textless{}-\ arrange(df,\ total)}

\subsubsection{\texorpdfstring{\textbf{\texttt{arrange(df,\ desc(column))}}}{arrange(df, desc(column))}}\label{arrangedf-desccolumn}

\textbf{Purpose:} Sorts rows in descending order.\\
\textbf{Example:}
\texttt{df\_sorted\ \textless{}-\ arrange(df,\ desc(total))}

\subsubsection{\texorpdfstring{\textbf{\texttt{summarize(df,\ new\_column\ =\ function(existing\_column))}}}{summarize(df, new\_column = function(existing\_column))}}\label{summarizedf-new_column-functionexisting_column}

\textbf{Purpose:} Aggregates data by computing summary statistics.\\
\textbf{Example:}
\texttt{df\_summary\ \textless{}-\ summarize(df,\ avg\_price\ =\ mean(unit\_price))}

\subsubsection{\texorpdfstring{\textbf{\texttt{group\_by(df,\ column)}}}{group\_by(df, column)}}\label{group_bydf-column}

\textbf{Purpose:} Groups data by a categorical variable before
summarization.\\
\textbf{Example:}
\texttt{df\_grouped\ \textless{}-\ group\_by(df,\ product\_line)}

\begin{center}\rule{0.5\linewidth}{0.5pt}\end{center}

\subsection{\texorpdfstring{\textbf{7. Chaining Commands
(Piping)}}{7. Chaining Commands (Piping)}}\label{chaining-commands-piping}

\subsubsection{\texorpdfstring{\textbf{\texttt{df\ \%\textgreater{}\%\ function()}}}{df \%\textgreater\% function()}}\label{df-function}

\textbf{Purpose:} Passes the result of one function directly to
another.\\
\textbf{Example:}

\begin{Shaded}
\begin{Highlighting}[]
\NormalTok{summary\_df }\OtherTok{\textless{}{-}}\NormalTok{ df }\SpecialCharTok{\%\textgreater{}\%}
  \FunctionTok{group\_by}\NormalTok{(product\_line) }\SpecialCharTok{\%\textgreater{}\%}
  \FunctionTok{summarize}\NormalTok{(}\AttributeTok{avg\_price =} \FunctionTok{mean}\NormalTok{(unit\_price))}
\end{Highlighting}
\end{Shaded}

\begin{center}\rule{0.5\linewidth}{0.5pt}\end{center}

\subsection{\texorpdfstring{\textbf{8. Appending and Combining
Data}}{8. Appending and Combining Data}}\label{appending-and-combining-data}

\subsubsection{\texorpdfstring{\textbf{\texttt{bind\_rows(df1,\ df2)}}}{bind\_rows(df1, df2)}}\label{bind_rowsdf1-df2}

\textbf{Purpose:} Appends two dataframes (stacks rows).\\
\textbf{Example:}
\texttt{df\_combined\ \textless{}-\ bind\_rows(df\_yangon,\ df\_mandalay)}

\begin{center}\rule{0.5\linewidth}{0.5pt}\end{center}

\subsection{\texorpdfstring{\textbf{9. Capturing and Running
Scripts}}{9. Capturing and Running Scripts}}\label{capturing-and-running-scripts}

\subsubsection{\texorpdfstring{\textbf{\texttt{sink("output.txt")}}}{sink("output.txt")}}\label{sinkoutput.txt}

\textbf{Purpose:} Redirects R console output to a text file.\\
\textbf{Example:} \texttt{sink("lab\_03\_log.txt")}

\subsubsection{\texorpdfstring{\textbf{\texttt{source("script.R")}}}{source("script.R")}}\label{sourcescript.r}

\textbf{Purpose:} Runs an external R script.\\
\textbf{Example:} \texttt{source("lab\_03.R")}

\begin{center}\rule{0.5\linewidth}{0.5pt}\end{center}

\subsubsection{\texorpdfstring{\textbf{Using This
Glossary}}{Using This Glossary}}\label{using-this-glossary}

\begin{itemize}
\tightlist
\item
  Reference this list while working through Lab 03.
\item
  Experiment with each function in R to solidify understanding.
\item
  Combine multiple functions using pipes (\texttt{\%\textgreater{}\%})
  to streamline analysis.
\end{itemize}

By understanding these core functions, you'll be able to efficiently
clean, manipulate, and analyze data in R!

\end{document}
