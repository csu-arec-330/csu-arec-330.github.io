% Options for packages loaded elsewhere
\PassOptionsToPackage{unicode}{hyperref}
\PassOptionsToPackage{hyphens}{url}
%
\documentclass[
  11pt,
]{article}
\usepackage{amsmath,amssymb}
\usepackage{iftex}
\ifPDFTeX
  \usepackage[T1]{fontenc}
  \usepackage[utf8]{inputenc}
  \usepackage{textcomp} % provide euro and other symbols
\else % if luatex or xetex
  \usepackage{unicode-math} % this also loads fontspec
  \defaultfontfeatures{Scale=MatchLowercase}
  \defaultfontfeatures[\rmfamily]{Ligatures=TeX,Scale=1}
\fi
\usepackage{lmodern}
\ifPDFTeX\else
  % xetex/luatex font selection
\fi
% Use upquote if available, for straight quotes in verbatim environments
\IfFileExists{upquote.sty}{\usepackage{upquote}}{}
\IfFileExists{microtype.sty}{% use microtype if available
  \usepackage[]{microtype}
  \UseMicrotypeSet[protrusion]{basicmath} % disable protrusion for tt fonts
}{}
\makeatletter
\@ifundefined{KOMAClassName}{% if non-KOMA class
  \IfFileExists{parskip.sty}{%
    \usepackage{parskip}
  }{% else
    \setlength{\parindent}{0pt}
    \setlength{\parskip}{6pt plus 2pt minus 1pt}}
}{% if KOMA class
  \KOMAoptions{parskip=half}}
\makeatother
\usepackage{xcolor}
\usepackage[margin=1in]{geometry}
\usepackage{longtable,booktabs,array}
\usepackage{calc} % for calculating minipage widths
% Correct order of tables after \paragraph or \subparagraph
\usepackage{etoolbox}
\makeatletter
\patchcmd\longtable{\par}{\if@noskipsec\mbox{}\fi\par}{}{}
\makeatother
% Allow footnotes in longtable head/foot
\IfFileExists{footnotehyper.sty}{\usepackage{footnotehyper}}{\usepackage{footnote}}
\makesavenoteenv{longtable}
\usepackage{graphicx}
\makeatletter
\def\maxwidth{\ifdim\Gin@nat@width>\linewidth\linewidth\else\Gin@nat@width\fi}
\def\maxheight{\ifdim\Gin@nat@height>\textheight\textheight\else\Gin@nat@height\fi}
\makeatother
% Scale images if necessary, so that they will not overflow the page
% margins by default, and it is still possible to overwrite the defaults
% using explicit options in \includegraphics[width, height, ...]{}
\setkeys{Gin}{width=\maxwidth,height=\maxheight,keepaspectratio}
% Set default figure placement to htbp
\makeatletter
\def\fps@figure{htbp}
\makeatother
\setlength{\emergencystretch}{3em} % prevent overfull lines
\providecommand{\tightlist}{%
  \setlength{\itemsep}{0pt}\setlength{\parskip}{0pt}}
\setcounter{secnumdepth}{-\maxdimen} % remove section numbering
\usepackage{setspace}
\setlength{\parindent}{0pt}
\ifLuaTeX
  \usepackage{selnolig}  % disable illegal ligatures
\fi
\usepackage{bookmark}
\IfFileExists{xurl.sty}{\usepackage{xurl}}{} % add URL line breaks if available
\urlstyle{same}
\hypersetup{
  pdftitle={Guidelines for Data Storytelling and Creating Effective Visualizations},
  hidelinks,
  pdfcreator={LaTeX via pandoc}}

\title{Guidelines for Data Storytelling and Creating Effective
Visualizations}
\author{}
\date{\vspace{-2.5em}}

\begin{document}
\maketitle

\section{Six Elements of a Data
Story}\label{six-elements-of-a-data-story}

\subsection{\texorpdfstring{\textbf{1. Data
Foundation}}{1. Data Foundation}}\label{data-foundation}

Your story is grounded in \textbf{real, observable data}. A strong data
foundation provides evidence for your narrative and helps build
credibility.

\textbf{Guiding Questions:}

\begin{itemize}
\tightlist
\item
  What data supports your story?
\item
  Is the data reliable, accurate, and relevant?
\end{itemize}

\subsection{\texorpdfstring{\textbf{2. Main
Point}}{2. Main Point}}\label{main-point}

Every story needs a \textbf{clear main point}. What message do you want
your audience to take away?

\textbf{Guiding Questions:}

\begin{itemize}
\tightlist
\item
  What is the central insight or argument?
\item
  Why does this matter to your audience?
\end{itemize}

\subsection{\texorpdfstring{\textbf{3. Explanatory
Focus}}{3. Explanatory Focus}}\label{explanatory-focus}

Your goal is to \textbf{explain} an insight, not just explore data. Your
story should help the audience understand \textbf{what is happening, who
is involved, and why it matters}.

\textbf{Guiding Questions:}

\begin{itemize}
\tightlist
\item
  What is the key takeaway?
\item
  Who are the main actors or groups affected?
\item
  What mechanisms explain the observed patterns?
\end{itemize}

\subsection{\texorpdfstring{\textbf{4. Linear
Sequence}}{4. Linear Sequence}}\label{linear-sequence}

A strong data story follows a \textbf{logical progression}, with
supporting points building toward the main takeaway.

\textbf{Example Sequence:}

\begin{itemize}
\tightlist
\item
  \textbf{Step 1:} Consumer preferences are shifting toward plant-based
  foods.\\
\item
  \textbf{Step 2:} A company introduces more plant-based products.\\
\item
  \textbf{Step 3:} Sales for plant-based products increase.
\end{itemize}

\textbf{Guiding Questions:}

\begin{itemize}
\tightlist
\item
  Are your supporting points structured logically?
\item
  Does each section build toward the main point?
\end{itemize}

\subsection{\texorpdfstring{\textbf{5. Dramatic
Elements}}{5. Dramatic Elements}}\label{dramatic-elements}

Data stories can be \textbf{humanized} by including \textbf{characters,
settings, and context}. Numbers alone don't create impact---your
audience needs to understand the \textbf{why} behind the data.

\textbf{Example:}

\begin{itemize}
\tightlist
\item
  ``Our store's annual sales hit \$200K last year.'' (Is this good or
  bad?)\\
\item
  ``Sales grew from \$100K to \$200K in just one year, doubling revenue
  after launching an online platform.'' (Now the context is clear!)
\end{itemize}

\textbf{Guiding Questions:}

\begin{itemize}
\tightlist
\item
  What details make the data relatable?
\item
  How can you frame the data in a way that resonates with your audience?
\end{itemize}

\subsection{\texorpdfstring{\textbf{6. Visual
Anchors}}{6. Visual Anchors}}\label{visual-anchors}

Visuals \textbf{enhance comprehension} and make key insights stand out.
Well-designed \textbf{charts, graphs, and images} should reinforce your
message.

\textbf{Guiding Questions:}

\begin{itemize}
\tightlist
\item
  Which visual best represents the main point?
\item
  Are the visuals simple, clear, and aligned with the story?
\end{itemize}

\subsubsection{\texorpdfstring{\textbf{Applying these
elements}}{Applying these elements}}\label{applying-these-elements}

When working on your D\textsuperscript{3}M projects, outline how each of
these six elements will shape your presentation. Focus on:

\begin{itemize}
\tightlist
\item
  Defining a clear main point
\item
  Organizing supporting evidence in a logical sequence
\item
  Using effective visuals to drive the message home
\end{itemize}

\begin{center}\rule{0.5\linewidth}{0.5pt}\end{center}

\section{Guidelines for Effective Data
Visualizations}\label{guidelines-for-effective-data-visualizations}

\subsection{Data Story vs.~Data
Visualization}\label{data-story-vs.-data-visualization}

A \textbf{data story} provides \textbf{context and explanation}, using a
sequence of insights to guide the audience toward a conclusion. A
\textbf{data visualization}, on the other hand, is a \textbf{single
representation} of data that helps the audience interpret trends,
relationships, or distributions. A strong data story integrates multiple
visualizations into a cohesive narrative.

\subsection{\texorpdfstring{\textbf{Six Guidelines for Better Data
Visualizations}}{Six Guidelines for Better Data Visualizations}}\label{six-guidelines-for-better-data-visualizations}

\subsubsection{\texorpdfstring{\textbf{1. Identify Your
Objective}}{1. Identify Your Objective}}\label{identify-your-objective}

Before designing your visualization, clarify:

\begin{itemize}
\tightlist
\item
  \textbf{What message do I want to convey?}
\item
  \textbf{What action should my audience take based on this
  information?}
\end{itemize}

A well-defined objective ensures your visualization serves its intended
purpose.

\subsubsection{\texorpdfstring{\textbf{2. Show the
Data}}{2. Show the Data}}\label{show-the-data}

Your audience can only grasp your point if they can \textbf{see} the
data clearly.

\begin{itemize}
\tightlist
\item
  Make sure that data is visible and not obscured by unnecessary
  elements.
\item
  Choose the appropriate chart type for your message.
\item
  Provide meaningful labels and scales.
\end{itemize}

\subsubsection{\texorpdfstring{\textbf{3. Reduce the
Clutter}}{3. Reduce the Clutter}}\label{reduce-the-clutter}

Unnecessary elements distract from the main message. Minimize:

\begin{itemize}
\tightlist
\item
  Heavy tick marks and grid lines.
\item
  Overlapping and overwhelming data markers.
\item
  Gradient colors, patterns, and excessive dimensions.
\item
  Excessive text, legends, or labels.
\end{itemize}

\textbf{Simpler = Better.} A clean, direct visualization improves
comprehension.

\subsubsection{\texorpdfstring{\textbf{4. Integrate Graphics and
Text}}{4. Integrate Graphics and Text}}\label{integrate-graphics-and-text}

Text enhance understanding.

\begin{itemize}
\tightlist
\item
  \textbf{Remove legends} when possible and label data directly.
\item
  \textbf{Use active titles} that summarize key takeaways.
\item
  \textbf{Add annotations} to highlight critical insights.
\end{itemize}

\subsubsection{\texorpdfstring{\textbf{5. Avoid the Spaghetti
Chart}}{5. Avoid the Spaghetti Chart}}\label{avoid-the-spaghetti-chart}

Overloaded line charts with too many overlapping lines create confusion.

\begin{itemize}
\tightlist
\item
  Simplify by showing only the most relevant information.
\item
  Break complex charts into smaller, focused components.
\item
  Highlight only the most important trends or comparisons.
\end{itemize}

\subsubsection{\texorpdfstring{\textbf{6. Start with
Gray}}{6. Start with Gray}}\label{start-with-gray}

To emphasize key insights strategically:

\begin{itemize}
\tightlist
\item
  Start with all elements in \textbf{gray}.
\item
  Gradually add \textbf{color} only where emphasis is needed.
\item
  This forces intentional use of visual cues to guide the audience's
  attention.
\end{itemize}

\subsection{\texorpdfstring{\textbf{Applying the
Guidelines}}{Applying the Guidelines}}\label{applying-the-guidelines}

Choose a past visualization you've created and evaluate it using these
six guidelines. Identify:

\begin{itemize}
\tightlist
\item
  \textbf{One area to simplify}
\item
  \textbf{One way to enhance clarity}
\item
  \textbf{One way to better integrate graphics and text}
\end{itemize}

\begin{center}\rule{0.5\linewidth}{0.5pt}\end{center}

\section{\texorpdfstring{Connecting D\textsuperscript{3}M, Data
Storytelling, and Effective
Visualizations}{Connecting D3M, Data Storytelling, and Effective Visualizations}}\label{connecting-d3m-data-storytelling-and-effective-visualizations}

This table outlines the key elements of the \textbf{Data-Driven
Decision-Making (D3M) process}, \textbf{data storytelling principles},
and \textbf{effective visualization techniques} to create clear,
evidence-based, actionable insights.

\begin{longtable}[]{@{}
  >{\raggedright\arraybackslash}p{(\columnwidth - 4\tabcolsep) * \real{0.2612}}
  >{\raggedright\arraybackslash}p{(\columnwidth - 4\tabcolsep) * \real{0.3582}}
  >{\raggedright\arraybackslash}p{(\columnwidth - 4\tabcolsep) * \real{0.3806}}@{}}
\toprule\noalign{}
\begin{minipage}[b]{\linewidth}\raggedright
\textbf{D3M Process}
\end{minipage} & \begin{minipage}[b]{\linewidth}\raggedright
\textbf{Data Storytelling}
\end{minipage} & \begin{minipage}[b]{\linewidth}\raggedright
\textbf{Effective Visualizations}
\end{minipage} \\
\midrule\noalign{}
\endhead
\bottomrule\noalign{}
\endlastfoot
\textbf{1. Define an Objective} & Identify the key message or insight. &
Choose the right chart type for the message. \\
\textbf{2. Establish a Hypothesis} & Frame a compelling narrative or
question. & Highlight expected vs.~unexpected results. \\
\textbf{3. Collect Relevant Data} & Use data to support the story's
credibility. & Use clean, reliable data sources. \\
\textbf{4. Analyze the Data} & Find patterns, relationships, and key
takeaways. & Use comparisons, trends, and breakdowns to show
insights. \\
\textbf{5. Interpret the Results} & Provide context: Who? What? Why? &
Remove clutter and focus attention on insights. \\
\textbf{6. Communicate Insights} & Deliver a clear, engaging message. &
Integrate text, annotations, and active titles. \\
\end{longtable}

\subsection{\texorpdfstring{\textbf{Main
Takeaways}}{Main Takeaways}}\label{main-takeaways}

\begin{itemize}
\tightlist
\item
  The \textbf{D\textsuperscript{3}M process} presents a structured
  approach to analyzing data.
\item
  \textbf{Data storytelling} makes insights meaningful by adding context
  and a narrative.
\item
  \textbf{Effective visualizations} enhance comprehension by reducing
  noise and focusing attention on key takeaways.
\end{itemize}

\subsubsection{\texorpdfstring{\textbf{Connecting the Three
Frameworks}}{Connecting the Three Frameworks}}\label{connecting-the-three-frameworks}

In preparing your projects, remember the following:

\begin{enumerate}
\def\labelenumi{\arabic{enumi}.}
\tightlist
\item
  Apply \textbf{D3M} to structure your analysis.
\item
  Use \textbf{data storytelling} techniques to craft a compelling
  narrative.
\item
  Implement \textbf{effective visualization} principles to communicate
  insights clearly.
\end{enumerate}

\subsection{\texorpdfstring{\textbf{Interconnected Framework: D3M, Data
Storytelling, and Effective
Visualizations}}{Interconnected Framework: D3M, Data Storytelling, and Effective Visualizations}}\label{interconnected-framework-d3m-data-storytelling-and-effective-visualizations}

\subsubsection{\texorpdfstring{\textbf{D3M: The Analytical
Approach}}{D3M: The Analytical Approach}}\label{d3m-the-analytical-approach}

\begin{itemize}
\tightlist
\item
  Guides how data is structured, analyzed, and interpreted.
\item
  Focuses on accuracy and objectivity.
\item
  Forms the foundation of evidence-based storytelling.
\end{itemize}

\textbf{Feedback Loop:} Analytical insights may evolve based on the
effectiveness of storytelling and visualization.

\begin{itemize}
\tightlist
\item
  Define the key message based on data analysis.
\item
  Identify the most important trends, comparisons, or relationships.
\item
  Frame findings in a way that highlights their significance.
\end{itemize}

\textbf{Feedback Loop:} Refining the story can highlight gaps in
analysis, requiring further D3M refinement.

\subsubsection{\texorpdfstring{\textbf{Data Storytelling: Communicating
the
Analysis}}{Data Storytelling: Communicating the Analysis}}\label{data-storytelling-communicating-the-analysis}

\begin{itemize}
\tightlist
\item
  Translates complex data into a structured, compelling narrative.
\item
  Aligns with principles of clarity, engagement, and relevance.
\item
  Ensures the audience understands the significance of the data.
\end{itemize}

\textbf{Feedback Loop:} Testing different narratives may reveal the need
for stronger visual representation or refined analysis.

\begin{itemize}
\tightlist
\item
  Craft a narrative that provides context and meaning.
\item
  Structure information in a logical, engaging sequence.
\item
  Emphasize key takeaways with relatable examples or analogies.
\end{itemize}

\textbf{Feedback Loop:} Reviewing visual effectiveness may lead to
adjustments in storytelling or analytical focus.

\subsubsection{\texorpdfstring{\textbf{Effective Data Visualizations:
Principles for
Clarity}}{Effective Data Visualizations: Principles for Clarity}}\label{effective-data-visualizations-principles-for-clarity}

\begin{itemize}
\tightlist
\item
  Apply design principles to reinforce the narrative.
\item
  Use appropriate chart types, color, and annotations to emphasize
  insights.
\item
  Reduce clutter while maintaining necessary context.
\end{itemize}

\textbf{Final Feedback Loop:} Visualizations should be clear and
impactful, supporting both the storytelling and analytical approach.

\begin{itemize}
\tightlist
\item
  Select the best chart type to support the narrative.
\item
  Highlight insights using color, annotations, and active titles.
\item
  Remove clutter to keep the focus on what matters most.
\end{itemize}

\subsection{\texorpdfstring{\textbf{The Continuous
Process}}{The Continuous Process}}\label{the-continuous-process}

\begin{itemize}
\tightlist
\item
  \textbf{D3M leads to clear insights}, which need a structured
  narrative.
\item
  \textbf{Data storytelling provides meaning}, making the information
  engaging and relatable.
\item
  \textbf{Effective visualizations enhance clarity}, allowing the
  audience to quickly grasp key messages.
\end{itemize}

\end{document}
